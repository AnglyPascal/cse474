\documentclass[article, 11pt, a4paper, oneside]{memoir}
\usepackage[article, noindex, serif]{ahsan}

% TODO: make a new command to set the dimensions of the margin, use ifthen

\title{\vspace{-3.5em}\Large\textbf{Midterm Paper Review}\\[1em] \LARGE \textbf{Social
Percolation Models} \\ \Large Solomon, S., Weisbuch, G., Arcangelis, L., Jan, N.,
Stauffer, D., 1999} 

\author{M Ahsan Al Mahir\\ \small BuX ID: 20216007\\ \small GSuite: ahsan.al.mahir@g.bracu.ac.bd}

\share

\begin{document}
\maketitle

The paper proposes a percolation model for simulating social recommendation situations.
The primary question it seeks to answer is ``Why do you notice extreme market shares,
close to 0\% and 100\% in media industry". Below we give a thorough review of the model
the authors proposed as a possible explanation to this question.


\begin{enumerate}[left=0pt, itemsep=20pt, label={\(\square\)}]

    \item \textbf{Which system is modeled and simulated in your selected paper? }

        To answer the proposed question, the authors came up with a Monte Carlo simulation
        based on the percolation theory, where people are considered as points on the
        lattice, and their neighbors are considered as their acquaintances. 

    \item \textbf{Why is this simulation needed? }

        Under usual assumptions, market share between different products should be
        uniformly distributed. But in reality that is not the case as many products are
        either hits or flops, and they do not follow an uniform distribution. The
        explanation the authors gave for this bi-modal distribution is that ``...certain
        commodities markets is often discussed in evolutionary economics in terms of
        bounded rationality..." (p. 240)

        The social percolation model they came up with presents a possible way to predict
        whether a product would be either a hit or a flop based on how close the quality
        of the product is to the percolation threshold of \(2D\) lattice. So by
        understanding their model, industries can gear their investment towards increasing
        their product quality based on the outcomes from the market.

    \item \textbf{What questions did the author ask to analyze the problem? }

        The authors noticed that most prior researches regarding propagation of new ideas
        throughout a social group considered the propagation in terms of epidemiology.
        They instead asked whether adoption of new ideas had any external effects or not.
        So they possibly had asked these questions

        \begin{enumerate}
            \item How does information travel when not all information is not public?
            \item How does prior knowledge about something influences the decision
                regarding some market product?
            \item How can social acquiantances influence the buyer's decision?
            \item If we think about product quality as a numeric value, then which values
                of the quality can make it a hit or a miss?
            \item How does the quality of a past product effect the preferences of a
                population?
            \item How should the quality of a product change after it experiences hits or
                flops?
        \end{enumerate}

    \item \textbf{Classify the model of the system. }

        The model is a discrete percolation model that uses Monte Carlo simulation.

    \item \textbf{Describe the abstract model the authors came up with. }

        The authors created a abstract structure for the social model by considering each
        person being a point on a square lattice, and acquiantances being lattice
        neighborhood. They also considered personal preferences and product quality as
        numbers between \([0, 1]\). They also took changes in personal preferences and
        product quality as numeric values into account to create a dynamic system.

        The main simulation of this system is to take an initial product value and to
        increment time step by step to find the minimum product value that would make the
        product a hit. 
        
        The authors selected a \(2D\) lattice to best represent social connections in a
        large city. But they conjectured that the result would be similar in other
        dimensions as well.

    \item \textbf{Describe the mathematical model model the authors came up with.  }

        The authors proposed two versions of the model, a simple static version to build
        the basis of the model, and then a dynamic model that takes changes in people's
        preferences and product quality in account.

        The static model has the following key properties:
        \begin{enumerate}[left=0pt, itemsep=0pt]
            \item The population would be points on a square lattice of size \(L\times
                L\). Each person will have a preference value \(0 \le p_i\le 1\). 
            \item The product quality will be a number \(0 \le q \le 1\).
            \item At first there will be a subset of the populations who will have the
                knowledge about the product. 
            \item At every increment of time, the current subset of the population who has
                knowledge about the product will decide whether to buy that product or
                not, by comparing their preference value with the product value. 

                That is, if \(p_i<q\) then they buy the product, if not then they don't.
            \item If a person buys the product, then they becomes a source of knowledge,
                and they let their neighbors on the lattice know about the product. So
                after the time increment, the neighbors of the buyers become knowledgeable
                about the product, and can decide whether to buy the product in the next
                time increment.
        \end{enumerate}

        A product is labeled a `hit' if after enough time increments, there is a spanning
        connected component in the lattice. And `flop' otherwise. Because of this, the
        authors claim that products that falls below the percolation threshold of 
        \(0.593\), it is a flop and if it is more than the threshold, then it is a hit.

        The authors improved this static model by taking into account that people's past
        experience with a product influences their preferences as well as the quality of a
        product changes when it experiences hits or flops. This dynamic model has the
        properties:
        \begin{enumerate}[left=0pt, itemsep=0pt]
            \item The system is updated after a product has been identified as a hit or a
                flop.
            \item During the update, every person who bought the previous product
                increases their preference value by \(\delta p\) because they were
                satisfied by their previous experience, and how have a higher expectation.
            \item The people who did not buy the product decreases their preference valu
                by \(\delta p \).
            \item If the product was a hit, then the production company decreases the
                product value by \(\delta q\) in attempt to minimize the production cost
                while maintaining their popularity.
            \item If the product was flop, then the production company increases the
                product value by \(\delta q\) to reach more audience.
        \end{enumerate}

        After several simulations of the system, the authors noted that the value \(q\)
        tends to the percolation threshold.


    \item \textbf{Describe the system state variables of this system.}

        The system state variables for this model are
        \begin{enumerate}[left=0pt, itemsep=0pt]
            \item The \(L \times L\) lattice which is stored in a two dimensional array.
            \item Personal preferences \(p_i\) for each of the people on the lattice.
            \item The initial product quality \(q\)
            \item The values \(\delta p, \delta q\) that the dynamic model uses.
        \end{enumerate}


    \item \textbf{Name the tools and programs the authors used for the simulation.}

        The authors did not mention any programs or tools in their paper. But assuming
        from their plots and simulation models, it can be inferred that their work can be
        reproduced using python libraries \texttt{numpy, scipy.ndimage, matplotlib.pyplot}
        and \texttt{random}.


    \item \textbf{What was the result of the simulation? What conclusions did the authors
        draw from their simulation output?}

        The authors concluded on a number of results. These can be listed in:
        \begin{enumerate}[left=0pt, itemsep=0pt]
            \item Firstly, the authors concluded that this percolaction model for social
                situations produces almost identical results as observed in reality. For
                example, the relationships between \(p_c\) and \(q\) can explain why some
                products experience hits and some experience flops.
            \item They also concluded that the ideal value of a product tends to the
                percoaltion value. Meaning that production companies can work with real
                life data to better predict their desired product value by using this
                model.
            \item Given enought time and for larger lattices, the value \(p_i\) and \(q\)
                approaches \(p_c=0.593\)
        \end{enumerate}


    \item \textbf{How did the authors verify and validate their model? Write your
        suggestions for improvement of the model.}

        The model the authors developed was intended as a possible explanation for a
        natural phenomena that cannot be explained by usual probabilistic methods. Their
        explanation for the possible solution matches their test data, that if \(q > p_c\)
        then the product will be a hit, and if \(q < p_c\) then it will be a miss. Which
        is intuitively correct based on results from percolation theory. 

        The results matching their initial conjectures verifies their model. To validate
        their model, they used the model with 8 different test systems, and related the
        results from those tests to other scientific results. Also their model
        although theoretically, but correctly explains the hit-flop phenomena. Which gives
        validation to their model.


    \item \textbf{Describe a new problem relevant to the paper you want to solve by
        modeling and simulation.}

        At the end of the lecture, the authors conjectured that this model can be used to estimate sales rate in products other than cinema as well. But they do not give explanation about whether this same model could be implemented for childrens' toys as in that scenerio, the neighborhood relationship between the children is not nearly as important as their parents'. 

        Estimating the sales of childrens' products poses a new set of complexity that arises when we take both the childrens' and their parents' preferences into account. In which case, the model might need heavy modification to correctly estimate market shares.

\end{enumerate}



\end{document}
