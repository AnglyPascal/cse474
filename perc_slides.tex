\documentclass[compress]{beamer}
\usepackage[beamer, ]{ahsan}
\usepackage{listings}
\usepackage{graphicx}

\title{Percolation}
\subtitle{A powerful tool in simulation}
\author{Ipshita Bonhi}
\date{\today}


\newcommand{\imp}[1]{\textcolor{NordRed}{#1}}
\lstset{
    numbers=left,
    xleftmargin=10pt,
    numbersep=8pt,
    numberstyle=\footnotesize\ttfamily\color{fgC!60},
    stepnumber=1,
    basicstyle=\linespread{1.1}\footnotesize\ttfamily,
    commentstyle=\color{fgC!60!pageC}\ttfamily,
    stringstyle=\small\color{NordBrightBlue},
    showstringspaces=false,
    breaklines=true,
    keywordstyle=\color{NordRed},
    morekeywords={def},
    emph={[2]return, plt, as, np},
    emphstyle={[2]\color{NordOrange}},
    emph={[3]imshow, show, rand},
    emphstyle={[3]\color{NordGreen}},
}

\begin{document}

\begin{frame}[plain,noframenumbering]
    \maketitle
\end{frame}

\color{NordBlack}

\section{What is it?}


\begin{frame}
    Consider a sponge as scene at the right. This kind of medium have tiny pores
    inside a solid skeleton. These kind of mediums are called ``Porous Medium''.

    \begin{center}
        \includegraphics[width=.4\linewidth]{Porousceramic.jpg}
    \end{center}

    Now suppose we have to measure the water flow through such materials. Since
    the pores inside a porous medium are placed completely randomly. 
\end{frame}

\begin{frame}

    So in order to study such random porous objects, we need a simulation method called
    \imp{Percolation}, which gives us to build a discrete version of these materials and
    study the connnected regions inside the medium.

    \vspace{12pt}

    This model also gives us the ability to monitor the spacous parts as well. It is most
    useful for physists and applied mathematicians. It also has applications to biology,
    computer science, and social sciences.

\end{frame}

\begin{frame}
    In a percolation model, we consider an \(L\times L\) box, which we call \imp{lattice}. 

    \vspace{12pt}
    We toss a biased coin with probability \(p\) for heads and probabilty \(1-p\) for
    tails to fill some of the cells up.

    \vspace{12pt}

    So for each of the cells, we toss a coin. If the coin lands on heads, then we color
    the cell black, which means it is filled, and otherwise keep it empty.
\end{frame}

\begin{frame}

    For example, the following three \(6\times 6\) lattices are randomly filled with
    probabilities \(p=.4, .5, .6\) respectively

    \vspace{15pt}

    \begin{minipage}{.3\linewidth}
        \includegraphics[width=\linewidth]{p1.png}
    \end{minipage}\hfill%
    \begin{minipage}{.3\linewidth}
        \includegraphics[width=\linewidth]{p2.png}
    \end{minipage}\hfill%
    \begin{minipage}{.3\linewidth}
        \includegraphics[width=\linewidth]{p3.png}
    \end{minipage}
\end{frame}

\begin{frame}
    \imp{Percolation is the study of connectivity.} 

    \vspace{12pt}

    In such a randomly generated lattice, we are interested in connected clusters.  A
    \imp{connected cluster} is made of some cells that are neighbor to each other and are
    all occupied. 
\end{frame}

\begin{frame}
    So in the first picture below, there are \(6\) connected clusters and there are
    \(7, \text{ and } 4\) in the second and third picture.

    \vspace{12pt}

    \begin{minipage}{.3\linewidth}
        \includegraphics[width=\linewidth]{p1.png}
    \end{minipage}\hfill%
    \begin{minipage}{.3\linewidth}
        \includegraphics[width=\linewidth]{p2.png}
    \end{minipage}\hfill%
    \begin{minipage}{.3\linewidth}
        \includegraphics[width=\linewidth]{p3.png}
    \end{minipage}
\end{frame}

\begin{frame}
    
    We are even more interested in connected clusters that spans from one side of the
    lattice to the other. We call them \imp{Spanning Components}. 

    \vspace{12pt}

    We can such systems where there is a spanning cluster \imp{Percolating system}.

\end{frame}

\begin{frame} 

    \begin{minipage}{.3\linewidth}
        For example, there is no spanning cluster that goes from top to bottom of the
        lattice in the first lattice below, but there is one in the second lattice.
    \end{minipage}\hfill%
    \begin{minipage}{.3\linewidth}
        \begin{center}
            \includegraphics[width=\linewidth]{spanning1.png}
        \end{center}
    \end{minipage}\hfill%
    \begin{minipage}{.3\linewidth}
        \begin{center}
            \includegraphics[width=\linewidth]{spanning2.png}
        \end{center}
    \end{minipage}

\end{frame}

\begin{frame}

    In short, we will ask this question:

    \vspace{12pt}

    \textcolor{NordOrange}{
        If we randomly generate a lattice with a biased coin with probability \(p\), for
        which values of \(0 \le p \le 1\) at system is almost certainly to be
        \textbf{Percolating}?\\
    }

    \vspace{12pt}

    \textcolor{NordOrange}{
        In other words, when will we almost certainly find a spanning connected cluster?
    }        

    \vspace{12pt}

    And in those scenarios, what happens to the connected clusters? 

\end{frame}

\begin{frame}
    \small

    Before we jump in to analyze percolation models and when such models are percolating,
    let's take a look at why do we even care about percolating theory beside studying
    porous materials.

    \begin{itemize}[left=0pt,label=\(\square\)]
        \item In situations where objects are linked to each other, and their properties
            effect others connected to them. For example in social marketing scenarios
            where buyers influence each others' preferences.
        \item In biological sciences, to predict the fragmentation of biological virus
            shells.
        \item In ecology to see how environment fragmentation impacts animal habitats.
        \item In multilayerd computer networking to find out how the layers interact with
            each other.
        \item To study traffic system in a city. Dynamic percolation models can predict
            the traffic capacity thresholds of particular junctions or roads.
        \item It has also found its application in discrete mathematics such as graph
            theory and different graph algorithms.
    \end{itemize}

    and many more...

    \normalsize
\end{frame}

\section{Taking a Closer Look}

\subsection{Basic structure}

\begin{frame}[fragile]
    So, how do we write a very basic version of this model? We want to create a lattice,
    generate a random number between \(0, 1\) for each of the cells, and check with the
    probability \(p\) if that cell should be occupied or not. The following simple code
    does such that.

    \vspace{10pt}

    \begin{lstlisting}[language=Python]
import matplotlib.pyplot as plt
import numpy.random as rnd

p = 0.5            # defining the probability here
l = rnd.rand(5, 5) # create a 8x8 array with a random probability
m = l < p          # checking if cells are occupied, true or false

plt.imshow(m, interpolation='None', cmap='binary')
plt.show()
    \end{lstlisting}
\end{frame}

\begin{frame}
    For example, one instance of this program would create the first figure shown below.
    On the second figure you can see the values of each random probability for each of the
    cells.

    \vspace{12pt}

    \begin{minipage}{.5\linewidth}
        \begin{center}
            \includegraphics[width=.7\linewidth]{random_array_true-false.png}
        \end{center}
    \end{minipage}\hfill%
    \begin{minipage}{.5\linewidth}
        \begin{center}
            \includegraphics[width=.74\linewidth]{random_array.png}
        \end{center}
    \end{minipage}
\end{frame}

\begin{frame}[fragile]
    Now since we are interested about the \imp{clusters}, we can have them in different
    shades to be like:

    \begin{minipage}{.63\linewidth}
        \begin{lstlisting}[language=python]
from scipy.ndimage import measurements
from random import shuffle

lw, num = measurements.label(m)
b = np.arange(lw.max() + 1)
shuffle(b[1:])
shuffledLw = b[lw]
        \end{lstlisting}
    \end{minipage}\hfill%
    \begin{minipage}{.35\linewidth}
        \begin{center}
        \includegraphics[width=\linewidth]{cluster.png}
        \end{center}
    \end{minipage}

\end{frame}

\subsection{Defining Percolation Threshold}

\begin{frame}
    We previously said that we will be interested at when we have a spanning cluster. To
    get a better understanding, we define the following:

    \vspace{12pt}

    \imp{\textbf{Percolation Threshold:}} \textcolor{NordOrange}{The smallest probability
    \(p\) for which any randomly generated lattice of a fixed size with probability \(p\)
    is almost certain to have a spanning cluster.}

\end{frame}

\begin{frame}
    For example, the percolation threshold for a 2D infinite lattice is \(0.593\) meaning
    that if we color the cells of an infinite 2D lattice with the probability \(p=0.593\),
    then there will be certainly a cluster that spans from negative infinity to positive
    infinity.
\end{frame}

\begin{frame}
    Now the natural question is, why is the threshold so important?

    \vspace{12pt}

    Say we have a random situation, where our product quality has to be a certain level to
    reach the most number of customers. Now we want to minimize the cost. So if we
    consider the situation to be a percolation model, our quality should be somewhere
    close to the percolation threshold.
\end{frame}

\subsection{Finding the Percolation Threshold}

\begin{frame}[fragile]
    Now, how do we find the percolation threshold for a given lattice of size \(L\times
    L\)? 
    \vspace{12pt}

    Previously where we used \texttt{lw, num = measurements.label(m)}, we named the
    clusters by numbers. For example, this like would turn the left \texttt{nparray} to
    the right \texttt{nparray}.

    \vspace{10pt}

    \begin{minipage}{.6\linewidth}
\begin{lstlisting}
[[ True  True False  True False]
 [ True False  True  True False]
 [False  True False False False]
 [ True  True False  True False]
 [ True False False  True False]]
\end{lstlisting}
    \end{minipage}\hfill%
    \begin{minipage}{.3\linewidth}
\begin{lstlisting}
[[1 1 0 2 0]
 [1 0 2 2 0]
 [0 3 0 0 0]
 [3 3 0 4 0]
 [3 0 0 4 0]]
\end{lstlisting}
    \end{minipage}
\end{frame}

\begin{frame}
    So if a spanning cluster existed, then the leftmost column and the rightmost column
    would have one number in common, which would mean that there were a cluster that
    spanned from the left border to the rightmost border.
\end{frame}

\begin{frame}[fragile]
    To find the percolation threshold, we divide the interval \([0, 1]\) into 100 or more
    equally spaced points. And then run the simulation to randomly generate lattices and
    check if there is spanning cluster in that lattice. The pseudocode is given below:

    \begin{lstlisting}
arr = np.linspace(0, 1, 100)
for p in arr:
    samples = produce_samples(p)
    if check_if_spanning_cluster_exists(samples):
        return p
    \end{lstlisting}

    This will find the smallest value of \(p\) for which almost certainly random lattices
    will be percolating.
\end{frame}


\section{Analyzing our simulation}

\begin{frame}
    To analyze the lattices, clusters and spanning clusters, we need to define a few
    things first.
\end{frame}

\subsection{Percolation Probability}

\begin{frame}

    We define \imp{\(\Pi(p, L)\)} to be the probability for there to be a cluster that
    spans from left to right of a lattice of size \(L\times L\) as a function of \(p, L\).
    We call it the \imp{percolation probability}.

    \vspace{1em}

    The next slide shows how \(\Pi(p, L)\) varies for different values of \(p\).
\end{frame}


\begin{frame}
    \begin{center}
        \includegraphics[width=\linewidth]{pLvsp.png}
    \end{center}

    The above diagram plots the values of \(\Pi(p, L)\) for \(L = 50, 100\) and \(200\).
    Notice how the percolation probabilty gets almost equal to \(1\) starting at the
    percolation threshold. 
\end{frame}


\subsection{Density of Spanning Cluster}

\begin{frame}
    We have studies the probability for there to be a spanning cluster. But what about the
    probability for each individual cell to be in a spanning cluster?

    \vspace{1em}

    We define \imp{\(P(p, L)\)} to be the probability for a cell to belong to a spanning
    cluster. We call it the \imp{density of the spanning cluster}

    \vspace{1em}

    The next page shows how the density of the spanning cluster varies with the value of
    \(p\).
\end{frame}

\begin{frame}
    \begin{center}
        \includegraphics[width=\linewidth]{ppLvsp.png}
    \end{center}

    The above diagram plots the values of \(P(p, L)\) for \(L = 50, 100\) and \(200\).
    Notice the steep rise in density at the percolation threshold. Also, the density is 1
    only when \(p=1\), as then all the cells will be occupied.
\end{frame}


\subsection{One Dimensional and Infinite Dimensional Lattice}

\begin{frame}
    As you might've already guess that solving a percolation problem, that is, knowing all
    the values of percolation threshold, probability or spanning cluster density
    deterministically is close to impossible.

    \vspace{1em}

    But for the one dimensional case (where the sites are arranged in a line) and the
    infinite dimensional case, the percolation problem is solvable, and easily analyzable.
\end{frame}

\subsubsection{One Dimension}
\begin{frame}
    The one dimensional lattice is just a straight line of lattice sites. For example, the
    following is a 1D lattice generated randomly with the probability \(p=.6\)

    \begin{center}
        \includegraphics[width=\linewidth]{1dlattice.png}
    \end{center}

    For the \textbf{one dimensional} lattice, the percolation threshold will be \(p_c =
    1\), because for there to be a spanning cluster from the negative infinity to the
    positive infinity, all of the sites need to be occupied.

\end{frame}

\begin{frame}
    In this case, we can also calculate the probability for a site to belong to a cluster
    of a specific size \(s\). Which is expressed as 
    \[P\left(\text{site belongs to a cluster of size } s\right)= s n(p, L)\] 
    where \(n(p, L)\) is called \imp{cluster number density}. Which is the probability
    that a site belongs to a specific location of a cluster of size \(s\).
\end{frame}

\begin{frame}
    Using the cluster number density, we can also calculate the average size of clusters
    in a one dimensional lattice. If we generate a random lattice with probability \(p\),
    then the \imp{Average Cluster Size} is given by 
    \[S(p) = \sum^{}_{s}s\left(\frac{sn(s, p)}{\sum^{}_{s} sn(s, p)}\right) \] 
    Which just counts the sum of expected lengths of all clusters of size \(s\).
\end{frame}

\subsubsection{Infinite Dimension}

\begin{frame}
    We can also solve the percolation problem in tree like structures where there are no
    loops. For example of one such tree, below is the \imp{Bethe Lattice}, also called as
    the \imp{Cayley Tree}, which is a tree structure where every node has \(Z\) neighbors.


    \begin{center}
        \includegraphics[width=.8\linewidth]{bethe.png}
    \end{center}

    In this tree, every node has \(3\) neighbors.
\end{frame}

\begin{frame}
    In such a tree, to have a spanning cluster, we must ensure that for each occupied node
    on that cluster to have one occupied neighbor. 

    \vspace{1em}

    If we start from the center of the tree and move along a branch, we will generate
    \(Z-1\) new neighbors every turn. So we just need to ensure that one of those
    neighbors are occupied. Because of this, our percolation threshold becomes

    \[p_c=\frac{1}{Z-1}\] 
\end{frame}



\end{document}
